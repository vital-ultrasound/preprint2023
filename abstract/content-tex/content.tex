%
\begin{keywords}
One, two, three, four, five
\end{keywords}
%

\section{Introduction}
\label{sec:intro}
In the last decades the use of Echochardiography is crucial in Intensive Care Units (ICU) advances of smaller US clinical devices, US image quality and functions and its real-time  capabilities to access cardiac anatomy and functions \cite{Feigenbaum1996, Vieillard-Baron2008, singh2007, cambell2018}.
However, despite the previous advances there is still challenges on finding standard views from experienced sonograpehrs that sometimes  such quantifcations are qualitative and subjective \cite{Feigenbaum1996}.

Assessing left ventricular ejection fraction (LVEF) is done at the point of care by clinicians with different expertise which is impacted on the rhythm and structural varations \cite{liu2021}.
However, automatic quantification of LVEF is still challenging at the point of care due to variation of protocols, skills levels \cite{field2011} and the nature of proving feedback on real-time \cite{liu2021}.

\section{Methods and materials}
Rank-2 non-negative matrix factorization \cite{yuan2017} and recently Robust Non-negative Matrix Factorization
\cite{dukler2018} are low-computation cost algorithms to automatic segment mitral valve.
Clustering techniques \cite{zhang2018} \cite{kusunose2021}.
The paper title (on the first page) should begin 1.38 inches (35 mm) from the
top edge of the page, centered, completely capitalized, and in Times 14-point,
boldface type.  The authors' name(s) and affiliation(s) appear below the title
in capital and lower case letters.  Papers with multiple authors and
affiliations may require two or more lines for this information.

\section{Results}
To achieve the best rendering both in the printed and digital proceedings, we
strongly encourage you to use Times-Roman font.  In addition, this will give
the proceedings a more uniform look.  Use a font that is no smaller than nine
point type throughout the paper, including figure captions.

In nine point type font, capital letters are 2 mm high.  If you use the
smallest point size, there should be no more than 3.2 lines/cm (8 lines/inch)
vertically.  This is a minimum spacing; 2.75 lines/cm (7 lines/inch) will make
the paper much more readable.  Larger type sizes require correspondingly larger
vertical spacing.  Please do not double-space your paper.  True-Type 1 fonts
are preferred.

The first paragraph in each section should not be indented, but all the
following paragraphs within the section should be indented as these paragraphs
demonstrate.

\section{Conclusions}
\label{sec:majhead}
Major headings, for example, "1. Introduction", should appear in all capital
letters, bold face if possible, centered in the column, with one blank line
before, and one blank line after. Use a period (".") after the heading number,
not a colon.

\subsection{Subheadings}
\label{ssec:subhead}

Subheadings should appear in lower case (initial word capitalized) in
boldface.  They should start at the left margin on a separate line.
 
\subsubsection{Sub-subheadings}
\label{sssec:subsubhead}

Sub-subheadings, as in this paragraph, are discouraged. However, if you
must use them, they should appear in lower case (initial word
capitalized) and start at the left margin on a separate line, with paragraph
text beginning on the following line.  They should be in italics.


\section{Illustrations, graphs, and photographs}
\label{sec:illust}

Illustrations must appear within the designated margins.  They may span the two
columns.  If possible, position illustrations at the top of columns, rather
than in the middle or at the bottom.  Caption and number every illustration.
All halftone illustrations must be clear black and white prints.  If you use
color, make sure that the color figures are clear when printed on a black-only
printer.

Since there are many ways, often incompatible, of including images (e.g., with
experimental results) in a \LaTeX\ document, below is an example of how to do
this %$\cite{Lamp86}.

% Below is an example of how to insert images. Delete the ``\vspace'' line,
% uncomment the preceding line ``\centerline...'' and replace ``imageX.ps''
% with a suitable PostScript file name.
% -------------------------------------------------------------------------
\begin{figure}[htb]

\begin{minipage}[b]{1.0\linewidth}
  \centering
  \centerline{\includegraphics[width=8.5cm]{example-image}}
%  \vspace{2.0cm}
  \centerline{(a) Result 1}\medskip
\end{minipage}
%
\begin{minipage}[b]{.48\linewidth}
  \centering
  \centerline{\includegraphics[width=4.0cm]{example-image}}
%  \vspace{1.5cm}
  \centerline{(b) Results 3}\medskip
\end{minipage}
\hfill
\begin{minipage}[b]{0.48\linewidth}
  \centering
  \centerline{\includegraphics[width=4.0cm]{example-image}}
%  \vspace{1.5cm}
  \centerline{(c) Result 4}\medskip
\end{minipage}
%
\caption{Example of placing a figure with experimental results.}
\label{fig:res}
%
\end{figure}

% To start a new column (but not a new page) and help balance the last-page
% column length use \vfill\pagebreak.
% -------------------------------------------------------------------------
\vfill
\pagebreak


\section{Compliance with ethical standards}
``This study was performed in line with the principles of
      the Declaration of Helsinki. Approval was granted by the Ethics
      Committee of University B (Date.../No. ...).''

\section{Acknowledgments}
\label{sec:acknowledgments}

``This work was supported by […] (Grant numbers) and […]. Author X has served on advisory boards for Company Y.''
    ``Author X is partially funded by Y. Author Z is a Founder and Director for Company C.''
