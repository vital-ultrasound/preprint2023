\section{Introduction} \label{sec:intro}
In the last decades the use of echocardiography is a crucial clinical approach in Intensive Care Units (ICU) because of the advances of smaller US clinical devices, US image quality and its real-time capabilities to access cardiac anatomy \cite{Feigenbaum1996, Vieillard-Baron2008, singh2007, cambell2018}.
However, despite the previous advances there is still challenges on finding standard views from experienced sonograpehrs that sometimes such quantifcations are qualitative and subjective \cite{Feigenbaum1996}.
Similarly, automatic quantification of left ventricular ejection fraction (LVEF) is still challenging at the point of care due to variation of protocols, skills levels \cite{field2011} and the nature of proving feedback on real-time \cite{liu2021}.

\subsection{Image Quality Assessment}
\cite{labs2021_in_miua} considers chamber clarity, depth gain, on-axis attributes, apical foreshoredness.

\subsection{Clustering techniques}
Zhang et al. mentioned that 23 view classes from 7168 individually labeled videos that ware classified with a 13-layer CNN to then viewed with the use of t-Distributed Stochastic Neighbor Embedding \cite{zhang2018}.
Kusunose et al. mentioned that other authors have reached an acciracy of 91-94 for 15-view classification while their work mentioned a 98.1 accuracy for five-prederminted views \cite{kusunose2021}.

\subsection{Auto-encoders}
Laumer et al. proposed a novel autoencoder-based framework to learn human interpretable representation of cardiac cycles from cardiac ultrasound data \cite{laumer2020},


\subsection{Contrastive Learning}
Methods on Contrastive Learning apparently address the challenge of required labelled data to identify pathologies in the images of dectect certain cardiac views.
Recently, Chartsias et al. use contrastive learning to train imbalanced cardiac datasets and they compared a naive baseline model to achieve a F1 score of up to 26 \$% \cite{chartsias2021-ASMUS}
Saeed et al. recently investigated contrastive pretraining to improve the DeepLabV3 and UNET segmentation networks of cardiac structers in ultrasound imaging.
Authors showed comparable results with state-of-the-art fully supervised algorithms and presents better results compared to EchoNet-Dynamic and CAMUS \cite{saeed2021MIDL}

\subsection{Others}
Rank-2 non-negative matrix factorization \cite{yuan2017} to generate End-Systole and End-Diastole for apical 4 view.  
Recently Robust Non-negative Matrix Factorization seems to be implement low-computation cost algorithms to automatic segment mitral valve \cite{dukler2018}.


\section{Methods and materials}
